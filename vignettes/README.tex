\emph{seasonal} is an easy-to-use and full-featured R-interface to
X-13ARIMA-SEATS, the newest seasonal adjustment software developed by
the \href{http://www.census.gov/srd/www/x13as/}{United States Census
Bureau}. X-13ARIMA-SEATS combines and extends the capabilities of the
older X-12ARIMA (developed by the Census Bureau) and TRAMO-SEATS
(developed by the Bank of Spain).

\begin{figure}[htbp]
\centering
\includegraphics[width=\textwidth]{images/inspect.jpg}
\caption{Graphical user interface for X-13}
\end{figure}

If you are new to seasonal adjustment or X-13ARIMA-SEATS, the automated
procedures of \emph{seasonal} allow you to quickly produce good seasonal
adjustments of time series. Start with the
\hyperref[installation]{Installation} and
\hyperref[getting-started]{Getting started} section and skip the rest.
Alternatively, \texttt{demo(seas)} gives an overview of the package
functionality.

If you are familiar with X-13ARIMA-SEATS, you may benefit from the
flexible input and output structure of \emph{seasonal}. The package
allows you to use (almost) all commands of X-13ARIMA-SEATS, and it can
import (almost) all output generated by X-13ARIMA-SEATS. The only
exception is the `composite' spec, which is easy to replicate in basic
R. Read the \hyperref[input]{Input} and \hyperref[output]{Output}
sections and have a look at the
\href{http://www.seasonal.website/examples.html}{website of seasonal},
where the examples from the official X-13ARIMA-SEATS
\href{http://www.census.gov/ts/x13as/docX13ASHTML.pdf}{manual} are
reproduced in R.

\emph{seasonal} includes a \hyperref[inspect]{graphical user interface}
that facitlitates the use of X-13 both for beginners and advanced users.
The final sections of this vignette cover additional topics:
\hyperref[chinese-new-ux5cux2520year-ux5cux2520indian-diwali-in-other-customized-holidays]{User
defined holidays}, such as Chinese New Year, the
\hyperref[production-use]{use of seasonal for production}, and the
\hyperref[import-x-13-models-and-series]{import of existing X-13 model
specs} to R.

\hyperdef{}{installation}{\section{Installation}\label{installation}}

Since version 1.2, \emph{seasonal} relies on the
\href{https://cran.r-project.org/package=x13binary}{x13binary} package
to access prebuilt binaries of X-13ARIMA-SEATS. To install both
packages, type to the R console:

\begin{verbatim}
install.packages("seasonal")      
\end{verbatim}

This automatically installs \emph{x13binary}. If you are using an older
version of R (\textless{} 3.2) on Windows, you have to install
\emph{x13binary} from source:

\begin{verbatim}
install.packages("x13binary", type = "source") 
\end{verbatim}

See the documentation of \texttt{?seasonal} if you want to set the path
to X-13 manually.

\hyperdef{}{getting-started}{\section{Getting
started}\label{getting-started}}

seas is the core function of the \emph{seasonal} package. By default,
seas calls the automatic procedures of X-13ARIMA-SEATS to perform a
seasonal adjustment that works well in most circumstances:

\begin{verbatim}
m <- seas(AirPassengers)
 
\end{verbatim}

The first argument of \texttt{seas} has to be a time series of class
\texttt{"ts"}. The function returns an object of class \texttt{"seas"}
that contains all necessary information on the adjustment.

There are several functions and methods for \texttt{"seas"} objects: The
\texttt{final} function returns the adjusted series, the \texttt{plot}
method shows a plot with the unadjusted and the adjusted series. The
\texttt{summary} method allows you to display an overview of the model:

\begin{verbatim}
final(m)
plot(m)
summary(m)
\end{verbatim}

By default, \texttt{seas} calls the SEATS adjustment procedure. If you
prefer the X11 adjustment procedure, use the following option (see the
\hyperref[input]{Input} section for details on how to use arbitrary
options with X-13):

\begin{verbatim}
seas(AirPassengers, x11 = "")
\end{verbatim}

A default call to \texttt{seas} also invokes the following automatic
procedures of X -13ARIMA-SEATS:

\begin{itemize}
\itemsep1pt\parskip0pt\parsep0pt
\item
  Transformation selection (log / no log)
\item
  Detection of trading day and Easter effects
\item
  Outlier detection
\item
  ARIMA model search
\end{itemize}

Alternatively, all inputs may be entered manually, as in the following
example:

\begin{verbatim}
seas(x = AirPassengers, 
     regression.variables = c("td1coef", "easter[1]", "ao1951.May"), 
     arima.model = "(0 1 1)(0 1 1)", 
     regression.aictest = NULL,
     outlier = NULL, 
     transform.function = "log")
\end{verbatim}

The \texttt{static} command returns the manual call of a model. The call
above can be easily generated from the automatic model:

\begin{verbatim}
static(m)
static(m, coef = TRUE)  # also fixes the coefficients
\end{verbatim}

If you have \emph{Shiny} installed, the \texttt{inspect} command offers
an easy way to analyze and modify a seasonal adjustment procedure (see
the section below for details):

\begin{verbatim}
inspect(m)
\end{verbatim}

\hyperdef{}{input}{\section{Input}\label{input}}

In \emph{seasonal}, it is possible to use almost the complete syntax of
X-13ARIMA- SEATS. This is done via the \texttt{...} argument in the
\texttt{seas} function. The X -13ARIMA-SEATS syntax uses \emph{specs}
and \emph{arguments}, with each spec optionally containing some
arguments. These spec-argument combinations can be added to
\texttt{seas} by separating the spec and the argument by a dot
(\texttt{.}). For example, in order to set the `variables' argument of
the `regression' spec equal to \texttt{td} and \texttt{ao1999.jan}, the
input to \texttt{seas} looks like this:

\begin{verbatim}
m <- seas(AirPassengers, regression.variables = c("td", "ao1955.jan"))
\end{verbatim}

Note that R vectors may be used as an input. If a spec is added without
any arguments, the spec should be set equal to an empty string (or,
alternatively, to an empty list, as in early versions). Several defaults
of \texttt{seas} are empty strings, such as the default
\texttt{seats = ""}. See the help page (\texttt{?seas}) for more details
on the defaults. Note the difference between \texttt{""} (meaning the
spec is enabled but has no arguments) and \texttt{NULL} (meaning the
spec is disabled).

It is possible to manipulate almost all inputs to X-13ARIMA-SEATS in
this way. The best way to learn about the relationship between the
syntax of X-13ARIMA- SEATS and \emph{seasonal} is to study the
\href{http://www.seasonal.website/examples.html}{comprehensive list of
examples}. For instance, example 1 in section 7.1 from the
\href{http://www.census.gov/ts/x13as/docX13ASHTML.pdf}{manual},

\begin{verbatim}
series { title  =  "Quarterly Grape Harvest" start = 1950.1
       period =  4
       data  = (8997 9401 ... 11346) }
arima { model = (0 1 1) }
estimate { }
\end{verbatim}

translates to R in the following way:

\begin{verbatim}
seas(AirPassengers,
     x11 = ""),
     arima.model = "(0 1 1)"
)
\end{verbatim}

\texttt{seas} takes care of the `series' spec, and no input beside the
time series has to be provided. As \texttt{seas} uses the SEATS
procedure by default, the use of X11 has to be specified manually. When
the `x11' spec is added as an input (like above), the mutually exclusive
and default `seats' spec is automatically disabled. With
\texttt{arima.model}, an additional spec-argument is added to the input
of X-13ARIMA-SEATS. As the spec cannot be used in the same call as the
`automdl' spec, the latter is automatically disabled.

There are some mutually exclusive specs in X-13ARIMA-SEATS. If more than
one mutually exclusive spec is included in \texttt{seas}, specs are
overwritten according the following priority rules:

\begin{itemize}
\itemsep1pt\parskip0pt\parsep0pt
\item
  Model selection

  \begin{enumerate}
  \def\labelenumi{\arabic{enumi}.}
  \itemsep1pt\parskip0pt\parsep0pt
  \item
    \texttt{arima}
  \item
    \texttt{pickmdl}
  \item
    \texttt{automdl} (default)
  \end{enumerate}
\item
  Adjustment procedure

  \begin{enumerate}
  \def\labelenumi{\arabic{enumi}.}
  \itemsep1pt\parskip0pt\parsep0pt
  \item
    \texttt{x11}
  \item
    \texttt{seats} (default)
  \end{enumerate}
\end{itemize}

As an alternative to the \texttt{...} argument, spec-arguments can also
be supplied as a named list. This is useful for programming:

\begin{verbatim}
seas(list = list(x = AirPassengers, x11 = ""))
\end{verbatim}

\hyperdef{}{output}{\section{Output}\label{output}}

\emph{seasonal} has a flexible mechanism to read data from
X-13ARIMA-SEATS. With the \texttt{series} function, it is possible to
import almost all output that can be generated by X-13ARIMA-SEATS. For
example, the following command returns the forecasts of the ARIMA model
as a \texttt{"ts"} time series:

\begin{verbatim}
m <- seas(AirPassengers)
series(m, "forecast.forecasts")
\end{verbatim}

Because the \texttt{forecast.save = "forecasts"} argument has not been
specified in the model call, \texttt{series} re-evaluates the call with
the `forecast' spec enabled. It is also possible to return more than one
output table at the same time:

\begin{verbatim}
series(m, c("forecast.forecasts", "d1"))
\end{verbatim}

You can use either the unique short names of X-13 (such as \texttt{d1}),
or the the long names (such as \texttt{forecasts}). Because the long
table names are not unique, they need to be combined with the spec name
(\texttt{forecast}). See \texttt{?series} for a complete list of
options.

Note that re-evaluation doubles the overall computation time. If you
want to speed it up, you have to be explicit about the output in the
model call:

\begin{verbatim}
m <- seas(AirPassengers, forecast.save = "forecasts")
series(m, "forecast.forecasts")
\end{verbatim}

Some specs, like `slidingspans' and `history', are time consuming.
Re-evaluation allows you to separate these specs from the basic model
call:

\begin{verbatim}
m <- seas(AirPassengers)
series(m, "history.saestimates")
series(m, "slidingspans.sfspans")
\end{verbatim}

If you are using the HTML version of X-13, the \texttt{out} function
shows the content of the main output in the browser:

\begin{verbatim}
out(m)
\end{verbatim}

\section{Graphs}\label{graphs}

There are several graphical tools to analyze a \texttt{seas} model. The
main plot function draws the seasonally adjusted and unadjusted series,
as well as the outliers. Optionally, it also draws the trend of the
seasonal decomposition:

\begin{verbatim}
m <- seas(AirPassengers, regression.aictest = c("td", "easter"))
plot(m)
plot(m, trend = TRUE)
\end{verbatim}

The \texttt{monthplot} function allows for a monthwise plot (or
quarterwise, with the same function name) of the seasonal and the SI
component:

\begin{verbatim}
monthplot(m)
monthplot(m, choice = "irregular")
\end{verbatim}

Also, many standard R function can be used to analyze a \texttt{"seas"}
model:

\begin{verbatim}
pacf(resid(m))
spectrum(diff(resid(m)))
plot(density(resid(m)))
qqnorm(resid(m))
\end{verbatim}

The \texttt{identify} method can be used to select or deselect outliers
by point and click. Click several times to loop through different
outlier types.

\begin{verbatim}
identify(m)
\end{verbatim}

\hyperdef{}{inspect}{\section{Inspect}\label{inspect}}

The \texttt{inspect} function is a graphical tool for choosing a
seasonal adjustment model, using
\emph{\href{http://shiny.rstudio.com}{Shiny}}, with the same structure
as the \href{http://www.seasonal.website}{demo website of seasonal}. To
install the latest version of Shiny, type:

\begin{verbatim}
install.packages("shiny")
\end{verbatim}

The goal of \texttt{inspect} is to summarize all relevant options, plots
and statistics that should be usually considered. \texttt{inspect} uses
a \texttt{"seas"} object as its main argument:

\begin{verbatim}
inspect(m)
\end{verbatim}

Frequently used options can be modified using the drop down selectors in
the upper left panel. Each change will result in a re-estimation of the
seasonal adjustment model. The R-call, the output and the summary are
updated accordingly.

Alternatively, the R-Call can be modified manually in the lower left
panel. Press `Run Call' to re-estimate the model and to adjust the
option selectors, the output, and the summary. With the `Close and
Import' button, inspect is closed and the call is imported to R. The
`static' button substitutes automatic procedures by the automatically
chosen spec-argument options, in the same way as \texttt{static}.

The views in the upper right panel can be selected from the drop down
menu. The views can also be customized (see \texttt{?inspect} for
details)

The lower right panel shows the summary, as descibed in the help page of
\texttt{?summary.seas}. The `Full X-13 output' button opens the complete
output of X-13 in a separate tab or window.

\section{Chinese New Year, Indian Diwali and other customized
holidays}\label{chinese-new-year-indian-diwali-and-other-customized-holidays}

seasonal includes \texttt{genhol}, a function that makes it easy to
model user-defined holiday regression effects. \texttt{genhol} is an R
replacement for the equally named software by the Census Office; no
additional installation is required. The function uses an object of
class \texttt{"Date"} as its first argument, which specifies the
occurrence of the holiday.

In order to adjust Indian industrial production for Diwali effects, use,
e.g.,:

\begin{verbatim}
# variables included in seasonal
# iip: Indian industrial production
# cny, diwali, easter: dates of Chinese New Year, Indian Diwali and Easter

seas(iip, 
x11 = "",
xreg = genhol(diwali, start = 0, end = 0, center = "calendar"), 
regression.usertype = "holiday"
)
\end{verbatim}

For more examples, including Chinese New Year and complex pre- and
post-holiday adjustments, see \texttt{?genhol}.

\hyperdef{}{production-use}{\section{Production
use}\label{production-use}}

While \emph{seasonal} offers a quick way to adjust a time series in R,
it is equally suited for the recurring processing of potentially large
numbers of time series. There are two kind of seasonal adjustments in
production use:

\begin{enumerate}
\def\labelenumi{\arabic{enumi}.}
\itemsep1pt\parskip0pt\parsep0pt
\item
  a periodic application of an adjustment model to a time series
\item
  an automated adjustment to a large number of time series
\end{enumerate}

This section shows how both tasks can be accomplished with
\emph{seasonal} and basic R.

\subsection{Storing calls and batch
processing}\label{storing-calls-and-batch-processing}

\texttt{seas} calls are R objects of the standard class \texttt{"call"}.
Like any R object, calls can be stored in a list. In order to extract
the call of a \texttt{"seas"} object, you can access the \texttt{\$call}
element or extract the static call with \texttt{static()}. For example,

\begin{verbatim}
# two different models for two different time series
m1 <- seas(fdeaths, x11 = "")
m2 <- seas(mdeaths, x11 = "")

l <- list()
l$c1 <- static(m1)  # static call (with automated procedures substituted)
l$c2 <- m2$call     # original call
\end{verbatim}

The list can be stored and re-evaluated if new data becomes available:

\begin{verbatim}
ll <- lapply(l, eval)
\end{verbatim}

which returns another list containing the re-evaluated \texttt{"seas"}
objects. If you want to extract the final series, use:

\begin{verbatim}
do.call(cbind, lapply(ll, final))
\end{verbatim}

Of course, you also can extract any other series, e.g.:

\begin{verbatim}
# seasonal component of an X11 adjustment, see ?series
do.call(cbind, lapply(ll, series, "d10"))
\end{verbatim}

\subsection{Automated adjustment of multiple
series}\label{automated-adjustment-of-multiple-series}

X-13 can also be applied to a large number of series, using automated
adjustment methods. This can be accomplished with a loop or an apply
function. It is useful to wrap the call to \texttt{seas} in a
\texttt{try} statement; that way, an error will not break the execution.
You need to develop an error handling strategy for these cases: You can
either drop them, use them without adjustment or switch to a different
automated routine.

\begin{verbatim}
# collect data 
dta <- list(fdeaths = fdeaths, mdeaths = mdeaths)

# loop over dta
ll <- lapply(dta, function(e) try(seas(e, x11 = "")))

# list failing models
is.err <- sapply(ll, class) == "try-error"
ll[is.err]

# return final series of successful evaluations
do.call(cbind, lapply(ll[!is.err], final))
\end{verbatim}

If you have several cores and want to speed things up, the process is
well suited for parallelization:

\begin{verbatim}
# a list with 100 time series
largedta <- rep(list(AirPassengers), 100)

library(parallel)  # this is part of a standard R installation
\end{verbatim}

If you are on Windows or want to use cluster parallelization, use
\texttt{parLapply}:

\begin{verbatim}
# set up cluster
cl <- makeCluster(detectCores())

# load 'seasonal' for each node
clusterEvalQ(cl, library(seasonal))

# export data to each node
clusterExport(cl, varlist = "largedta")

# run in parallel (2.2s on a 8-core Macbook, vs 9.6s with standard lapply)
parLapply(cl, largedta, function(e) try(seas(e, x11 = "")))

# finally, stop the cluster
stopCluster(cl)
\end{verbatim}

On Linux or OS-X, `forking' parallelization allows you to do the same in
a single line:

\begin{verbatim}
mclapply(cl, largedta, function(e) try(seas(e, x11 = "")))
\end{verbatim}

\hyperdef{}{import-x-13-models-and-series}{\section{Import X-13 models
and series}\label{import-x-13-models-and-series}}

Two experimental utility functions allow you to import \texttt{.spc}
files and X-13 data files from any X-13 set-up. Simply locate the path
of your X-13 \texttt{.spc} file, and the \texttt{import.spc} function
will construct the corresponding call to \texttt{seas} as well as the
calls for importing the data.

\begin{verbatim}
# importing the orginal X-13 example file
import.spc(system.file("tests", "Testairline.spc", package="seasonal"))
\end{verbatim}

If data is stored outside the \texttt{.spc} file (as it usually will
be), the calls will make use of the \texttt{import.ts} function, which
imports arbitrary X-13 data files as R time series. See
\texttt{?import.ts} for examples.

\section{License and Credits}\label{license-and-credits}

\emph{seasonal} is free and open source, licensed under GPL-3. It
requires the X -13ARIMA-SEATS software by the U.S. Census Bureau, which
is open source and freely available under the terms of its own
\href{https://www.census.gov/srd/www/disclaimer.html}{license}.

\emph{seasonal} has been originally developed for the use at the Swiss
State Secretariat of Economic Affairs. It has been greatly improved over
time thanks to suggestions and support from Matthias Bannert, Freya
Beamish, Vidur Dhanda, Alain Galli, Ronald Indergand, Preetha
Kalambaden, Stefan Leist, James Livsey, Brian Monsell, Pinaki Mukherjee,
Bruno Parnisari, and many others. I am especially grateful to Dirk
Eddelbuettel for the fantastic work on the
\href{https://cran.r-project.org/package=x13binary}{x13binary} package.

Please report bugs and suggestions on
\href{https://github.com/christophsax/seasonal}{Github} or send me an
\href{mailto:christoph.sax@gmail.com}{e-mail}. Thank you!
